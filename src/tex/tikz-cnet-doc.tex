\documentclass[a4paper]{ltxdoc}
\usepackage{calc}
\usepackage{tikz}
\usepackage[a4paper,left=2.25cm,right=2.25cm,top=2.5cm,bottom=2.5cm,nohead]{geometry}
\usepackage[hidelinks]{hyperref}

\input{pgfmanual-en-macros.tex}

\usetikzlibrary{shapes.cnet}
\usetikzlibrary{shapes.cisco}
\usetikzlibrary{shapes.cnet.symbol.test}

\title{\tikz{\node[cnet car]{}}\\The \textsf{tikz-cnet} package}
\author{Paweł Tomulik}
\date{}

\begin{document}
\maketitle

\begin{abstract}
  This package provides \pgfname{} shapes that may be used to draw diagrams of
  computer networks. Several reusable |symbol|s are provided to simplify this
  task.
\end{abstract}

\section{Introduction}
\label{sec:intro}

\subsection{Usage}
\label{sec:usage}

The library may be included with |\usetikzlibrary{shapes.cnet}|:
\begin{codeexample}[preamble={\usetikzlibrary{shapes.cnet}}]
\begin{tikzpicture}
  \node[cnet car] {A car};
\end{tikzpicture}
\end{codeexample}
You can also limit inclusion to only necessary symbols with
|\usetikzlibrary{shapes.cnet.<symbol>}|, for example:
\begin{codeexample}[preamble={\usetikzlibrary{shapes.cnet.car}}]
\begin{tikzpicture}
  \node[cnet car] {A car};
\end{tikzpicture}
\end{codeexample}

\section{Symbols}\label{sec:symbols}

The |shapes.cnet| library defines
\begin{itemize}
  \item one generic |symbol|, named |cnet symbol|, which can be provided with
    custom \pgfname\ drawing code, and
  \item a~bunch of predefined specific |symbol|s with fixed drawings.
\end{itemize}

\subsection{Generic cnet symbol}
\label{sec:cnet-symbol}

The generic |cnet symbol| provides a basic node model, without any predefined
drawing. The model consists of two regions:
\begin{itemize}
  \item |picture|, and
  \item |text| region.
\end{itemize}
The |picture| encloses a~|drawing|, which is by default empty. The parameter
|/pgf/cnet/drawing| may be used to provide drawing. The |/pgf/cnet/drawing|
shall be a valid \pgfname{} code.

The |text| region contains the node contents as provided within brackets |{}|
of the |\node| command.

\begin{codeexample}[]
\begin{tikzpicture}
  \def\mydrawing{
    \pgfpathmoveto{\pgfpoint{-1.5cm}{-1.5cm}}
    \pgfpathlineto{\pgfpoint{1.5cm}{1.5cm}}
    \pgfusepath{stroke}
    \pgfpathcircle{\pgfpoint{0cm}{0cm}}{4pt}
    \pgfusepath{stroke,fill}
    \pgftext[top,at={\pgfpoint{0cm}{-0.60cm}}]{drawing}
  }
  \node[
    cnet symbol,
    draw=green, dashed, very thick,
    cnet/drawing size=3cm,
    cnet/picture outline=blue,
    cnet/picture outline dashed,
    cnet/picture outline very thin,
    cnet/drawing color=red,
    cnet/drawing solid,
    cnet/drawing={\mydrawing}
  ] {text};
\end{tikzpicture}
\end{codeexample}

\subsection{Common parameters}
\label{sec:cnet-symbol-parameters}

A~number of parameters can be used to setup the generic |cnet symbol|. Many of
them (but not all) also alter predefined symbols described in further chapters.

\begin{key}{/pgf/cnet/picture inner xsep=\meta{dimension} (initially .3333em)}
\keyalias{tikz/cnet}
  Stores the horizontal inner separation between the drawing and the border of
  picture region.
\end{key}

\begin{key}{/pgf/cnet/picture inner ysep=\meta{dimension} (initially .3333em)}
\keyalias{tikz/cnet}
  Stores the vertical inner separation between the drawing and the border of
  picture region.
\end{key}

\begin{key}{/pgf/cnet/picture inner sep=\meta{dimension}}
\keyalias{tikz/cnet}
  This style sets both |/pgf/cnet/picture inner xsep| and
  |/pgf/cnet/picture inner ysep| to \meta{dimension}.
\begin{codeexample}[]
\begin{tikzpicture}
  \tikzset{square/.style={
    shape=cnet symbol,
    cnet/drawing size=2cm,
    cnet/picture outline=black,
    cnet/picture outline dashed,
    cnet/picture outline very thin,
    cnet/drawing color=red,
    cnet/drawing fill color=lightgray,
    cnet/drawing={
      % rectangle spans entire drawing region
      \pgfpathrectangle{\pgfpoint{-1cm}{-1cm}}{\pgfpoint{2cm}{2cm}}
      \pgfusepath{stroke,fill}
    }
  }}
  \node[square, cnet/picture inner sep=.5cm] at (0cm,1.5cm) {};
  \node[square, cnet/picture inner sep=.1cm] at (0cm,-1.5cm) {};
\end{tikzpicture}
\end{codeexample}
\end{key}

\begin{key}{/pgf/cnet/picture outer xsep=\meta{dimension} (initially .5\string\pgflinewidth)}
\keyalias{tikz/cnet}
  Stores the recommended horizontal separation between the border of picture
  and the ,,outer anchors''.
\end{key}

\begin{key}{/pgf/cnet/picture outer ysep=\meta{dimension} (initially .5\string\pgflinewidth)}
\keyalias{tikz/cnet}
  Stores the recommended vertical separation between the border of picture
  and the ,,outer anchors''.
\end{key}

\begin{key}{/pgf/cnet/picture outer sep}
\keyalias{tikz/cnet}
  Sets both, |/pgf/cnet/picture outer xsep|
  and |/pgf/cnet/picture outer ysep| to \meta{dimension}.
\end{key}

\begin{key}{/pgf/cnet/picture minimum width=\meta{dimension} (initially 1pt)}
\keyalias{tikz/cnet}
  Stores the recommended minimum width of the picture region.
\end{key}

\begin{key}{/pgf/cnet/picture minimum height=\meta{dimension} (initially 1pt)}
\keyalias{tikz/cnet}
  Stores the recommended minimum height of the picture region.
\end{key}

\begin{key}{/pgf/cnet/picture minimum size=\meta{dimension}}
\keyalias{tikz/cnet}
  Sets both, |/pgf/cnet/picture minimum width|
  and |/pgf/cnet/picture minimum height| to \meta{dimension}.
\end{key}

\begin{key}{/pgf/cnet/picture text sep=\meta{dimension} (initially .3333em)}
\keyalias{tikz/cnet}
  Stores the vertical separation between text and picture.
\end{key}

\begin{key}{/pgf/cnet/picture outline=\meta{color}}
\keyalias{tikz/cnet}
  Causes the picture border to be drawn. If the optional \meta{color} argument
  is given, drawing is done using the given \meta{color}. Otherwise, color
  provided with |/tikz/draw| or |\textbackslash{color}| will be used.
\end{key}

\begin{key}{/pgf/cnet/picture fill=\meta{color}}
\keyalias{tikz/cnet}
  Cauese the picture region to be filled with \meta{color}
\end{key}

\begin{key}{/pgf/cnet/picture outline line width=\meta{dimension}}
\keyalias{tikz/cnet}
  Stores the line width used to draw the picture outline.
\end{key}

\begin{key}{/pgf/cnet/picture outline dash=\meta{dash pattern}phase\meta{dash phase}}
\keyalias{tikz/cnet}
  Sets the dashing pattern and phase for picture outline at the same time.
\end{key}

\begin{key}{/pgf/cnet/picture outline dash pattern=\meta{dash pattern}}
\keyalias{tikz/cnet}
  Sets the dashing pattern for picture outline. The syntax is the same as in
  |METAFONT|. For example following pattern of |2pt off 3pt on 4pt off 4pt|
  means ``draw 2pt, then leave out 3pt, then draw 4pt once more, then leave out
  4pt again, repeat''.
\end{key}

\begin{key}{/pgf/cnet/picture outline dash phase=\meta{dash phase}}
\keyalias{tikz/cnet}
  Shifts the start of the dash pattern for picture outline by \meta{dash phase}.
\end{key}

\begin{key}{/pgf/cnet/drawing=\meta{drawing code}}
\keyalias{tikz/cnet}
  Provides the drawing code for |cnet symbol|.
\end{key}

\begin{key}{/pgf/cnet/drawing width=\meta{dimension} (initially 3em)}
\keyalias{tikz/cnet}
  Stores the width of the drawing region.
\end{key}

\begin{key}{/pgf/cnet/drawing height=\meta{dimension} (initially 3em)}
\keyalias{tikz/cnet}
  Stores the height of the drawing region.
\end{key}

\begin{key}{/pgf/cnet/drawing height=\meta{dimension}}
\keyalias{tikz/cnet}
  Sets both |/pgf/cnet/drawing width| and |/pgf/cnet/drawing height| to
  \meta{dimension}.
\end{key}

\begin{key}{/pgf/cnet/drawing color=\meta{color}}
\keyalias{tikz/cnet}
  Sets stroke color used by drawing.
\end{key}

\begin{key}{/pgf/cnet/drawing fill color=\meta{color}}
\keyalias{tikz/cnet}
  Sets fill color used by drawing.
\end{key}

\begin{key}{/pgf/cnet/drawing line width=\meta{dimension}}
\keyalias{tikz/cnet}
  Sets line width used by drawing.
\end{key}

\begin{key}{/pgf/cnet/drawing dash=\meta{dash pattern}phase\meta{dash phase}}
\keyalias{tikz/cnet}
  Sets dash pattern and phase to be used by drawing.
\end{key}

\begin{key}{/pgf/cnet/drawing dash pattern=\meta{dash pattern}}
\keyalias{tikz/cnet}
  Sets the dashing pattern for drawing. The syntax is the same as in
  |METAFONT|. For example following pattern of |2pt off 3pt on 4pt off 4pt|
  means ``draw 2pt, then leave out 3pt, then draw 4pt once more, then leave out
  4pt again, repeat''.
\end{key}

\begin{key}{/pgf/cnet/drawing dash phase=\meta{dash phase}}
\keyalias{tikz/cnet}
  Shifts the start of the dash pattern for drawing by \meta{dash phase}.
\end{key}

\begin{key}{/pgf/cnet/anchor border=\meta{type}}
\keyalias{tikz/cnet}
  Decides what should be used as node's |anchor border|. Possible choices for
  \meta{type} are:
  \begin{itemize}
    \item |picture border| the boundaries of symbol's picture are used,
    \item |symbol border| the boundaries of whole symbol are used,
    \item |text border| the boundaries of text label are used.
  \end{itemize}

\begin{codeexample}[]
\begin{tikzpicture}
  \tikzset{circ/.style={
    shape=cnet symbol,
    draw=green, dashed, very thin,
    cnet/drawing size=1cm,
    cnet/picture outline=black,
    cnet/picture outline dashed,
    cnet/picture outline very thin,
    cnet/drawing solid,
    cnet/drawing color=red,
    cnet/drawing fill color=lightgray,
    cnet/drawing={
      \pgfpathcircle{\pgfpoint{0cm}{0cm}}{0.35cm}
      \pgfusepath{stroke,fill}
    }
  }}
  \node[circ, cnet/anchor border=picture border] (A) at (0cm,1cm) {A};
  \node[circ, cnet/anchor border=symbol border] (B) at (-1cm,-1cm) {B};
  \node[circ, cnet/anchor border=text border] (C) at (1cm,-1cm) {C};
  \draw (A) -- (B) -- (C);
\end{tikzpicture}
\end{codeexample}
\end{key}

\begin{key}{/pgf/cnet/text placement=\meta{placement} (default below picture)}
\keyalias{tikz/cnet}
  Defines placement of text relative to picture region. The available choices
  are:
  \begin{itemize}
    \item |above picture| -- the text will be placed above picture,
    \item |below picture| -- the text will be placed below picture.
  \end{itemize}

\begin{codeexample}[]
\begin{tikzpicture}
  \tikzset{circ/.style={
    shape=cnet symbol,
    cnet/drawing size=2cm,
    cnet/picture outline=black,
    cnet/picture outline dashed,
    cnet/picture outline very thin,
    cnet/drawing color=red,
    cnet/drawing fill color=lightgray,
    cnet/drawing={
      \pgfpathcircle{\pgfpoint{0cm}{0cm}}{0.75cm}
      \pgfusepath{stroke,fill}
    }
  }}
  \node[circ, cnet/text placement=above picture] at (0cm,1.5cm) {above picture};
  \node[circ, cnet/text placement=below picture] at (0cm,-1.5cm) {below picture};
\end{tikzpicture}
\end{codeexample}
\end{key}

\subsection{Common predefined styles}
\label{sec:cnet-symbol-predefined-styles}

\begin{stylekey}{/tikz/cnet/picture outline ultra thin}
  Sets |/tikz/cnet/picture outline line width| to |0.1pt|
\begin{codeexample}[]
\begin{tikzpicture}
  \node[
    cnet symbol,
    cnet/picture outline,
    cnet/picture outline ultra thin
  ] {};
\end{tikzpicture}
\end{codeexample}
\end{stylekey}

\begin{stylekey}{/tikz/cnet/picture outline very thin}
  Sets |/tikz/cnet/picture outline line width| to |0.2pt|
\begin{codeexample}[]
\begin{tikzpicture}
  \node[
    cnet symbol,
    cnet/picture outline,
    cnet/picture outline very thin
  ] {};
\end{tikzpicture}
\end{codeexample}
\end{stylekey}

\begin{stylekey}{/tikz/cnet/picture outline thin}
  Sets |/tikz/cnet/picture outline line width| to |0.4pt|
\begin{codeexample}[]
\begin{tikzpicture}
  \node[
    cnet symbol,
    cnet/picture outline,
    cnet/picture outline thin
  ] {};
\end{tikzpicture}
\end{codeexample}
\end{stylekey}

\begin{stylekey}{/tikz/cnet/picture outline semithick}
  Sets |/tikz/cnet/picture outline line width| to |0.6pt|
\begin{codeexample}[]
\begin{tikzpicture}
  \node[
    cnet symbol,
    cnet/picture outline,
    cnet/picture outline semithick
  ] {};
\end{tikzpicture}
\end{codeexample}
\end{stylekey}

\begin{stylekey}{/tikz/cnet/picture outline thick}
  Sets |/tikz/cnet/picture outline line width| to |0.8pt|
\begin{codeexample}[]
\begin{tikzpicture}
  \node[
    cnet symbol,
    cnet/picture outline,
    cnet/picture outline thick
  ] {};
\end{tikzpicture}
\end{codeexample}
\end{stylekey}

\begin{stylekey}{/tikz/cnet/picture outline very thick}
 Sets |/tikz/cnet/picture outline line width| to |1.2pt|
\begin{codeexample}[]
\begin{tikzpicture}
  \node[
    cnet symbol,
    cnet/picture outline,
    cnet/picture outline very thick
  ] {};
\end{tikzpicture}
\end{codeexample}
\end{stylekey}

\begin{stylekey}{/tikz/cnet/picture outline ultra thick}
  Sets |/tikz/cnet/picture outline line width| to |1.6pt|
\begin{codeexample}[]
\begin{tikzpicture}
  \node[
    cnet symbol,
    cnet/picture outline,
    cnet/picture outline ultra thick
  ] {};
\end{tikzpicture}
\end{codeexample}
\end{stylekey}


\begin{stylekey}{/tikz/cnet/picture outline solid}
  Sets |/tikz/cnet/picture outline dash pattern| to |{}| (empty, i.e. -- use
  solid lines).
\begin{codeexample}[]
\begin{tikzpicture}
  \tikzset{
    circ/.style={
      shape=cnet symbol, dashed,
      cnet/picture outline,
      cnet/drawing={
        \pgfpathcircle{\pgfpoint{0em}{0em}}{1em}
        \pgfusepath{stroke}
      }
    }
  }
  \node[circ] at (0em,3em) {without solid};
  \node[circ, cnet/picture outline solid] at (0em,-3em) {with solid};
\end{tikzpicture}
\end{codeexample}
\end{stylekey}

\begin{stylekey}{/tikz/cnet/picture outline dotted}
  Sets |/tikz/cnet/picture outline dash pattern| to dotted pattern.
\begin{codeexample}[]
\begin{tikzpicture}
  \tikzset{
    circ/.style={
      shape=cnet symbol, dashed,
      cnet/picture outline,
      cnet/drawing={
        \pgfpathcircle{\pgfpoint{0em}{0em}}{1em}
        \pgfusepath{stroke}
      }
    }
  }
  \node[circ] at (0em,3em) {without dotted};
  \node[circ, cnet/picture outline dotted] at (0em,-3em) {with dotted};
\end{tikzpicture}
\end{codeexample}
\end{stylekey}

\begin{stylekey}{/tikz/cnet/picture outline densely dotted}
  Sets |/tikz/cnet/picture outline dash pattern| to densely dotted pattern.
\begin{codeexample}[]
\begin{tikzpicture}
  \tikzset{
    circ/.style={
      shape=cnet symbol, dashed,
      cnet/picture outline,
      cnet/drawing={
        \pgfpathcircle{\pgfpoint{0em}{0em}}{1em}
        \pgfusepath{stroke}
      }
    }
  }
  \node[circ] at (0em,3em) {without dotted};
  \node[circ, cnet/picture outline densely dotted] at (0em,-3em) {with dotted};
\end{tikzpicture}
\end{codeexample}
\end{stylekey}

\begin{stylekey}{/tikz/cnet/picture outline loosely dotted}
  Sets |/tikz/cnet/picture outline dash pattern| to |on \pgflinewidth off 4pt|
\end{stylekey}

\begin{stylekey}{/tikz/cnet/picture outline dashed}
  Sets |/tikz/cnet/picture outline dash pattern| to |on 3pt off 3pt|
\end{stylekey}

\begin{stylekey}{/tikz/cnet/picture outline densely dashed}
  Sets |/tikz/cnet/picture outline dash pattern| to |on 3pt off 2pt|
\end{stylekey}

\begin{stylekey}{/tikz/cnet/picture outline loosely dashed}
  Sets |/tikz/cnet/picture outline dash pattern| to |on 3pt off 6pt|
\end{stylekey}

\begin{stylekey}{/tikz/cnet/picture outline dashdotted}
  Sets |/tikz/cnet/picture outline dash pattern| to |on 3pt off 2pt on \the\pgflinewidth off 2pt|
\end{stylekey}

\begin{stylekey}{/tikz/cnet/picture outline dash dot}
  Sets |/tikz/cnet/picture outline dash pattern| to |on 3pt off 2pt on \the\pgflinewidth off 2pt|
\end{stylekey}

\begin{stylekey}{/tikz/cnet/picture outline densely dashdotted}
  Sets |/tikz/cnet/picture outline dash pattern| to |on 3pt off 1pt on \the\pgflinewidth off 1pt|
\end{stylekey}

\begin{stylekey}{/tikz/cnet/picture outline densely dash dot}
  Sets |/tikz/cnet/picture outline dash pattern| to |on 3pt off 1pt on \the\pgflinewidth off 1pt|
\end{stylekey}

\begin{stylekey}{/tikz/cnet/picture outline loosely dashdotted}
  Sets |/tikz/cnet/picture outline dash pattern| to |on 3pt off 4pt on \the\pgflinewidth off 4pt|%
\end{stylekey}

\begin{stylekey}{/tikz/cnet/picture outline loosely dash dot}
  Sets |/tikz/cnet/picture outline dash pattern| to |on 3pt off 4pt on \the\pgflinewidth off 4pt|
\end{stylekey}

\begin{stylekey}{/tikz/cnet/picture outline dashdotdotted}
  Sets |/tikz/cnet/picture outline dash pattern| to |on 3pt off 2pt on \the\pgflinewidth off 2pt on \the\pgflinewidth off 2pt|
\end{stylekey}

\begin{stylekey}{/tikz/cnet/picture outline densely dashdotdotted}
  Sets |/tikz/cnet/picture outline dash pattern| to |on 3pt off 1pt on \the\pgflinewidth off 1pt on \the\pgflinewidth off 1pt|
\end{stylekey}

\begin{stylekey}{/tikz/cnet/picture outline loosely dashdotdotted}
  Sets |/tikz/cnet/picture outline dash pattern| to |on 3pt off 4pt on \the\pgflinewidth off 4pt on \the\pgflinewidth off 4pt|
\end{stylekey}

\begin{stylekey}{/tikz/cnet/picture outline dash dot dot}
  Sets |/tikz/cnet/picture outline dash pattern| to |on 3pt off 2pt on \the\pgflinewidth off 2pt on \the\pgflinewidth off 2pt|
\end{stylekey}

\begin{stylekey}{/tikz/cnet/picture outline densely dash dot dot}
  Sets |/tikz/cnet/picture outline dash pattern| to |on 3pt off 1pt on \the\pgflinewidth off 1pt on \the\pgflinewidth off 1pt|
\end{stylekey}

\begin{stylekey}{/tikz/cnet/picture outline loosely dash dot dot}
  Sets |/tikz/cnet/picture outline dash pattern| to |on 3pt off 4pt on \the\pgflinewidth off 4pt on \the\pgflinewidth off 4pt|
\end{stylekey}


%%\begin{stylekey}{/tikz/cnet/drawing ultra thin}
%%% {cnet/drawing line width=0.1pt}}%
%%  TODO
%%\end{stylekey}
%%
%%\begin{stylekey}{/tikz/cnet/drawing very thin}
%%% {cnet/drawing line width=0.2pt}}%
%%  TODO
%%\end{stylekey}
%%
%%\begin{stylekey}{/tikz/cnet/drawing thin}
%%% {cnet/drawing line width=0.4pt}}%
%%  TODO
%%\end{stylekey}
%%
%%\begin{stylekey}{/tikz/cnet/drawing semithick}
%%% {cnet/drawing line width=0.6pt}}%
%%  TODO
%%\end{stylekey}
%%
%%\begin{stylekey}{/tikz/cnet/drawing thick}
%%% {cnet/drawing line width=0.8pt}}%
%%  TODO
%%\end{stylekey}
%%
%%\begin{stylekey}{/tikz/cnet/drawing very thick}
%%% {cnet/drawing line width=1.2pt}}%
%%  TODO
%%\end{stylekey}
%%
%%\begin{stylekey}{/tikz/cnet/drawing ultra thick}
%%% {cnet/drawing line width=1.6pt}}%
%%  TODO
%%\end{stylekey}
%%
%%
%%\begin{stylekey}{/tikz/cnet/drawing solid}
%%% {cnet/drawing dash pattern=}}%
%%  TODO
%%\end{stylekey}
%%
%%\begin{stylekey}{/tikz/cnet/drawing dotted}
%%% {cnet/drawing dash pattern=on \pgflinewidth off 2pt}}%
%%  TODO
%%\end{stylekey}
%%
%%\begin{stylekey}{/tikz/cnet/drawing densely dotted}
%%% {cnet/drawing dash pattern=on \pgflinewidth off 1pt}}%
%%  TODO
%%\end{stylekey}
%%
%%\begin{stylekey}{/tikz/cnet/drawing loosely dotted}
%%% {cnet/drawing dash pattern=on \pgflinewidth off 4pt}}%
%%  TODO
%%\end{stylekey}
%%
%%\begin{stylekey}{/tikz/cnet/drawing dashed}
%%% {cnet/drawing dash pattern=on 3pt off 3pt}}%
%%  TODO
%%\end{stylekey}
%%
%%\begin{stylekey}{/tikz/cnet/drawing densely dashed}
%%% {cnet/drawing dash pattern=on 3pt off 2pt}}%
%%  TODO
%%\end{stylekey}
%%
%%\begin{stylekey}{/tikz/cnet/drawing loosely dashed}
%%% {cnet/drawing dash pattern=on 3pt off 6pt}}%
%%  TODO
%%\end{stylekey}
%%
%%\begin{stylekey}{/tikz/cnet/drawing dashdotted}
%%% {cnet/drawing dash pattern=on 3pt off 2pt on \the\pgflinewidth off 2pt}}%
%%  TODO
%%\end{stylekey}
%%
%%\begin{stylekey}{/tikz/cnet/drawing dash dot}
%%% {cnet/drawing dash pattern=on 3pt off 2pt on \the\pgflinewidth off 2pt}}%
%%  TODO
%%\end{stylekey}
%%
%%\begin{stylekey}{/tikz/cnet/drawing densely dashdotted}
%%% {cnet/drawing dash pattern=on 3pt off 1pt on \the\pgflinewidth off 1pt}}%
%%  TODO
%%\end{stylekey}
%%
%%\begin{stylekey}{/tikz/cnet/drawing densely dash dot}
%%% {cnet/drawing dash pattern=on 3pt off 1pt on \the\pgflinewidth off 1pt}}%
%%  TODO
%%\end{stylekey}
%%
%%\begin{stylekey}{/tikz/cnet/drawing loosely dashdotted}
%%% {cnet/drawing dash pattern=on 3pt off 4pt on \the\pgflinewidth off 4pt}}%
%%  TODO
%%\end{stylekey}
%%
%%\begin{stylekey}{/tikz/cnet/drawing loosely dash dot}
%%% {cnet/drawing dash pattern=on 3pt off 4pt on \the\pgflinewidth off 4pt}}%
%%  TODO
%%\end{stylekey}
%%
%%\begin{stylekey}{/tikz/cnet/drawing dashdotdotted}
%%% {cnet/drawing dash pattern=on 3pt off 2pt on \the\pgflinewidth off 2pt on \the\pgflinewidth off 2pt}}%
%%  TODO
%%\end{stylekey}
%%
%%\begin{stylekey}{/tikz/cnet/drawing densely dashdotdotted}
%%% {cnet/drawing dash pattern=on 3pt off 1pt on \the\pgflinewidth off 1pt on \the\pgflinewidth off 1pt}}%
%%  TODO
%%\end{stylekey}
%%
%%\begin{stylekey}{/tikz/cnet/drawing loosely dashdotdotted}
%%% {cnet/drawing dash pattern=on 3pt off 4pt on \the\pgflinewidth off 4pt on \the\pgflinewidth off 4pt}}%
%%  TODO
%%\end{stylekey}
%%
%%\begin{stylekey}{/tikz/cnet/drawing dash dot dot}
%%% {cnet/drawing dash pattern=on 3pt off 2pt on \the\pgflinewidth off 2pt on \the\pgflinewidth off 2pt}}%
%%  TODO
%%\end{stylekey}
%%
%%\begin{stylekey}{/tikz/cnet/drawing densely dash dot dot}
%%% {cnet/drawing dash pattern=on 3pt off 1pt on \the\pgflinewidth off 1pt on \the\pgflinewidth off 1pt}}%
%%  TODO
%%\end{stylekey}
%%
%%\begin{stylekey}{/tikz/cnet/drawing loosely dash dot dot}
%%% {cnet/drawing dash pattern=on 3pt off 4pt on \the\pgflinewidth off 4pt on \the\pgflinewidth off 4pt}}%
%%  TODO
%%\end{stylekey}


\subsection{Cnet symbol geometry}
\label{sec:cnet-symbol-geometry}

\autoref{fig:cnet-symbol-anchors} depicts standard anchors provided by
|cnet symbol|. The outer anchors, such as |north east|, are distanced from
symbol's boundary by |outer xsep| and |outer ysep|.
\begin{figure}[htbp]
  \begin{center}
  \begin{tikzpicture}
    \pic[] (A) at (0,0) {cnet symbol test anchors};
  \end{tikzpicture}
  \end{center}
  \caption{Anchors of cnet symbol (without picture anchors)}
  \label{fig:cnet-symbol-anchors}
\end{figure}

\autoref{fig:cnet-symbol-picture-anchors} depicts anchors related to
|cnet symbol|'s picture region. The outer anchors, such as |picture north east|,
are distanced from picture's boundary by |cnet picture outer xsep|
and~|cnet picture outer ysep|.
\begin{figure}[htbp]
  \begin{center}
  \begin{tikzpicture}
    \pic[] (A) at (0,0) {cnet symbol test picture anchors};
  \end{tikzpicture}
  \end{center}
  \caption{Picture anchors of cnet symbol}
  \label{fig:cnet-symbol-picture-anchors}
\end{figure}

\section{Predefined symbols}
\label{sec:predefined-symbols}

\begin{stylekey}{/tikz/cnet car}
Sets |shape={cnet car}|.
\begin{codeexample}[preamble={\usetikzlibrary{shapes.cnet.car}}]
\tikz{\node[cnet car] {};}
\end{codeexample}
\end{stylekey}

\begin{stylekey}{/tikz/cnet freezer}
Sets |shape={cnet freezer}|.
\begin{codeexample}[preamble={\usetikzlibrary{shapes.cnet.freezer}}]
\tikz{\node[cnet freezer] {};}
\end{codeexample}
\end{stylekey}

\begin{stylekey}{/tikz/cnet laptop}
Sets |shape={cnet laptop}|.
\begin{codeexample}[preamble={\usetikzlibrary{shapes.cnet.laptop}}]
\tikz{\node[cnet laptop] {};}
\end{codeexample}
\end{stylekey}

\begin{stylekey}{/tikz/cnet monitor}
Sets |shape={cnet monitor}|.
\begin{codeexample}[preamble={\usetikzlibrary{shapes.cnet.monitor}}]
\tikz{\node[cnet monitor] {};}
\end{codeexample}
\end{stylekey}

\begin{stylekey}{/tikz/cnet rf waves}
Sets |shape={cnet rf waves}|.
\begin{codeexample}[preamble={\usetikzlibrary{shapes.cnet.rfwaves}}]
\tikz{\node[cnet rf waves] {};}
\end{codeexample}
\end{stylekey}

\begin{stylekey}{/tikz/cnet server}
Sets |shape={cnet server}|.
\begin{codeexample}[preamble={\usetikzlibrary{shapes.cnet.server}}]
\tikz{\node[cnet server] {};}
\end{codeexample}
\end{stylekey}

\begin{stylekey}{/tikz/cnet smartphone}
Sets |shape={cnet smartphone}|.
\begin{codeexample}[preamble={\usetikzlibrary{shapes.cnet.smartphone}}]
\tikz{\node[cnet smartphone] {};}
\end{codeexample}
\end{stylekey}

\begin{stylekey}{/tikz/cnet thermometer}
Sets |shape={cnet thermometer}|.
\begin{codeexample}[preamble={\usetikzlibrary{shapes.cnet.thermometer}}]
\tikz{\node[cnet thermometer] {};}
\end{codeexample}
\end{stylekey}

\begin{stylekey}{/tikz/cnet traffic lights}
Sets |shape={cnet traffic lights}|.
\begin{codeexample}[preamble={\usetikzlibrary{shapes.cnet.trafficlights}}]
\tikz{\node[cnet traffic lights] {};}
\end{codeexample}
\end{stylekey}

\begin{stylekey}{/tikz/cnet wag54g}
Sets |shape={cnet wag54g}|.
\begin{codeexample}[preamble={\usetikzlibrary{shapes.cnet.wag54g}}]
\tikz{\node[cnet wag54g] {};}
\end{codeexample}
\end{stylekey}

\begin{stylekey}{/tikz/cnet workstation}
Sets |shape={cnet workstation}|.
\begin{codeexample}[preamble={\usetikzlibrary{shapes.cnet.workstation}}]
\tikz{\node[cnet workstation] {};}
\end{codeexample}
\end{stylekey}

\subsection{Combining symbols}
\label{sec: combining-symbols}

\begin{codeexample}[preamble={\usetikzlibrary{shapes.cnet.car}}]
\begin{tikzpicture}
  \node[cnet car] (S) {};
  \node[cnet rf waves] (W) at (S.picture north) {};
\end{tikzpicture}
\end{codeexample}

\begin{codeexample}[preamble={\usetikzlibrary{shapes.cnet.smartphone}}]
\begin{tikzpicture}
  \node[cnet smartphone] (S) {};
  \node[cnet rf waves] (W) at (S.picture north) {};
\end{tikzpicture}
\end{codeexample}


\section{Cisco network symbols}
\label{sec:cisco-network-symbols}

\begin{stylekey}{/tikz/cisco 100 BaseT hub}
Sets |shape={cisco 100 BaseT hub}|.
\begin{codeexample}[preamble={\usetikzlibrary{shapes.cisco.100basethub}}]
\tikz{\node[cisco 100 BaseT hub] {};}
\end{codeexample}
\end{stylekey}

\begin{stylekey}{/tikz/cisco cloud}
Sets |shape={cisco cloud}|.
\begin{codeexample}[preamble={\usetikzlibrary{shapes.cisco.cloud}}]
\tikz{\node[cisco cloud] {};}
\end{codeexample}
\end{stylekey}

\begin{stylekey}{/tikz/cisco firewall}
Sets |shape={cisco firewall}|.
\begin{codeexample}[preamble={\usetikzlibrary{shapes.cisco.firewall}}]
\tikz{\node[cisco firewall] {};}
\end{codeexample}
\end{stylekey}

\begin{stylekey}{/tikz/cisco key}
Sets |shape={cisco key}|.
\begin{codeexample}[preamble={\usetikzlibrary{shapes.cisco.key}}]
\tikz{\node[cisco key] {};}
\end{codeexample}
\end{stylekey}

\begin{stylekey}{/tikz/cisco laptop}
Sets |shape={cisco laptop}|.
\begin{codeexample}[preamble={\usetikzlibrary{shapes.cisco.laptop}}]
\tikz{\node[cisco laptop] {};}
\end{codeexample}
\end{stylekey}

\begin{stylekey}{/tikz/cisco macintosh}
Sets |shape={cisco macintosh}|.
\begin{codeexample}[preamble={\usetikzlibrary{shapes.cisco.macintosh}}]
\tikz{\node[cisco macintosh] {};}
\end{codeexample}
\end{stylekey}

\begin{stylekey}{/tikz/cisco mac woman}
Sets |shape={cisco mac woman}|.
\begin{codeexample}[preamble={\usetikzlibrary{shapes.cisco.macwoman}}]
\tikz{\node[cisco mac woman] {};}
\end{codeexample}
\end{stylekey}

\begin{stylekey}{/tikz/cisco man woman}
Sets |shape={cisco man woman}|.
\begin{codeexample}[preamble={\usetikzlibrary{shapes.cisco.manwoman}}]
\tikz{\node[cisco man woman] {};}
\end{codeexample}
\end{stylekey}

\begin{stylekey}{/tikz/cisco pc}
Sets |shape={cisco pc}|.
\begin{codeexample}[preamble={\usetikzlibrary{shapes.cisco.pc}}]
\tikz{\node[cisco pc] {};}
\end{codeexample}
\end{stylekey}

\begin{stylekey}{/tikz/cisco pc man}
Sets |shape={cisco pc man}|.
\begin{codeexample}[preamble={\usetikzlibrary{shapes.cisco.pcman}}]
\tikz{\node[cisco pc man] {};}
\end{codeexample}
\end{stylekey}

\begin{stylekey}{/tikz/cisco router}
Sets |shape={cisco router}|.
\begin{codeexample}[preamble={\usetikzlibrary{shapes.cisco.router}}]
\tikz{\node[cisco router] {};}
\end{codeexample}
\end{stylekey}

\begin{stylekey}{/tikz/cisco wireless router}
Sets |shape={cisco wireless router}|.
\begin{codeexample}[preamble={\usetikzlibrary{shapes.cisco.wirelessrouter}}]
\tikz{\node[cisco wireless router] {};}
\end{codeexample}
\end{stylekey}

\begin{stylekey}{/tikz/cisco workstation}
Sets |shape={cisco workstation}|.
\begin{codeexample}[preamble={\usetikzlibrary{shapes.cisco.workstation}}]
\tikz{\node[cisco workstation] {};}
\end{codeexample}
\end{stylekey}

\end{document}
